\documentclass[a4paper]{article}
\pagestyle{plain}
\usepackage[utf8]{inputenc}
\usepackage[english]{babel}
\usepackage[pdftex]{graphicx}
\usepackage{amsmath,amssymb}

\paperwidth=210mm
\paperheight=297mm

\hoffset=0.0mm 		% By default offset from paper edge is 1 inch, this will set it to 25mm
\voffset=-5.4mm		% By default offset from paper edge is 1 inch, this will set it to 20mm

\oddsidemargin=0mm	% This will add nothing to hoffset
\evensidemargin=0mm	% This will add nothing to hoffset
\topmargin=0mm
\headheight=0mm
\headsep=0mm

\textwidth=170mm
\textheight=237mm

\begin{document}
\title{Sakevych Michael}
\date{\vspace*{-6е}}
\maketitle
\begin{center} 
{\small{\tt sakevich2001@gmail.com}}
\end{center}
\textbf{Education}
\noindent\rule{\textwidth}{1pt}
\textbf{Technical Skills}
\noindent\rule{\textwidth}{1pt}
\begin{itemize}
	\item Ядро - містить генетичний матеріал (ДНК) і контролює діяльність клітини.
	\item Клітинна мембрана(глікокалікс) - гнучкий шар, який оточує клітину і контролює речовини, що надходять і виходять.
	\item Цитоплазма - желеподібна речовина, в якій відбуваються хімічні реакції.
	\item Мітохондрії - тут енергія виділяється з молекул їжі.
\end{itemize}
\textbf{Experience}
\begin{itemize}
	\item Клітинна стінка - Твердий шар поза клітинною мембраною, що містить целюлозу для забезпечення міцності рослини.
	\item Вакуоль - простір усередині клітини, який використовується для зберігання речовин і допомагає клітині зберігати форму.
	\item Хлоропласти - структури, що містять зелений пігмент хлорофіл, який є ключовою частиною фотосинтезу.
\end{itemize}
\textbf{Additional Courses}
\noindent\rule{\textwidth}{1pt}
\begin{table}[ht]
\centering
\caption{Основні відмінності між клітинами рослинного та тваринячого походження}\label{Table1}
\vspace*{1ex}
\begin{tabular}{|p{7cm}|p{7cm}|}
\hline
Клітини рослин & Клітини тварин \\
\hline
\multicolumn{2}{|c|}{Форма клітини}\\
\hline
Квадратні або прямокутні & Неправильна або кругла форма\\
\hline
\multicolumn{2}{|c|}{Клітинна стінка}\\
\hline
Містить & Немає\\
\hline
\multicolumn{2}{|c|}{Плазматична або клітинна мембрана}\\
\hline
Присутня & Присутня\\
\hline
\multicolumn{2}{|c|}{Ендоплазматичний ретикулум}\\
\hline
Присутня & Присутня\\
\hline
\multicolumn{2}{|c|}{Ядро}\\
\hline
Присутнє і лежить з одного боку клітини & Присутній і лежить у центрі клітини\\
\hline
\multicolumn{2}{|c|}{Лізосоми}\\
\hline
Присутні, але дуже рідкісні & Присутні \\
\hline
\multicolumn{2}{|c|}{Центросоми}\\
\hline
Відсутні & Присутні\\
\hline
\multicolumn{2}{|c|}{Апарат Гольджі}\\
\hline
Присутній & Присутній\\
\hline
\multicolumn{2}{|c|}{Цитоплазма}\\
\hline
Присутня & Присутня\\
\hline
\multicolumn{2}{|c|}{Рибосоми}\\
\hline
Присутні & Присутні\\
\hline
\multicolumn{2}{|c|}{Пластиди}\\
\hline
Присутні & Відсутні\\
\hline
\multicolumn{2}{|c|}{Вакуолі}\\
\hline
Одна  велика або декілька вакуоль, розташовані в центрі & Зазвичай маленькі і численні\\
\hline
\multicolumn{2}{|c|}{Джгутики}\\
\hline
Присутні у деяких & Присутні у більшості тваринних клітин\\
\hline
\multicolumn{2}{|c|}{Мітохондрії}\\
\hline
Присутні у невеликій кількості & Присутні у великій кількості \\
\hline
\multicolumn{2}{|c|}{Тип харчування}\\
\hline
Більшість автотрофний & Гетеротрофний \\
\hline
\end{tabular}
\end{table}
\begin{itemize}
	\item Рослини не можуть харчуватися, вживаючи їжу, тому їм потрібно виробляти поживні речовини іншим шляхом, наприклад шляхом поглинання сонячного світла. Цей 
	процес називається фотосинтез та відбувається в хлоропласті.
 	Після того, як цукор виготовлений, він потім розщеплюється мітохондріями для отримання енергії для клітини. Оскільки тварини отримують цукор з їжею, яку вони 
	їдять, їм не потрібні хлоропласти: просто мітохондрії.
	\item Як рослинні, так і тваринні клітини мають вакуолі. Рослинна клітина зазвичай містить велику одиничну вакуолю, яка використовується для зберігання та підтримки 
	форми клітини та використовується у якості резервуару для води. На противагу цьому, клітини тварин мають багато менших вакуолей.
	\item Рослинні клітини мають клітинну стінку, а також клітинну мембрану. У рослин клітинна стінка оточує клітинну мембрану. Це надає рослинній клітині унікальної
	прямокутної форми. Клітини тварин просто мають клітинну мембрану, але клітинної стінки немає.
\end{itemize}
\begin{itemize}
	\item Більшість клітин рослин можуть перетворюватись з одних на інші за потреби. В той же час з клітин тварин лише стовбурові клітини можуть перетворюватись. 
	\item Тваринні клітини, на відміну від рослинних, здатні генерувати лише 10 амінокислот з 20 (інші 10 вони отримують з їжі).
	\item Рослинні клітини зберігають енергію формуючи крохмал, а тваринні - утворюючи вуглецеві глюкогени.
	\item Рослинні клітини зазвичай розміром від 10 до 100 мікрометрів, а тваринні - 10-30.
	\item Рослинні клітини нерухомі (вийнятками є статеві клітини нижчих та вищих спорових рослин).
	\item Клітини тварин не мають пластид. Рослинні клітини містять такі пластиди, як хлоропласти, які необхідні для фотосинтезу .
	\item У всіх тварин клітин є центріолі, в той час як лише у деяких нижчих форм рослин є центріолі в клітинах;
	\item Рослини не здатні активно пересуватися, тому пристосувалися автотрофне способу харчування, синтезуючи самостійно всі необхідні поживні 
	речовини з неорганічних сполук.
	\item Взаємозв'язок між клітинами рослин здійснюється за рахунок плазмодесми, тварин - за допомогою десмосом.
\end{itemize}
\textbf{Achievements and Extracurricular Activities}
\noindent\rule{\textwidth}{1pt}
\end{document}